\section{Background}
\label{sec:background}

Data race detection is probably one of the most widely studied problems in
concurrent programs design and has been shown to be
NP-hard~\cite{netzer-miller}.
%
Data races are in general the symptoms for a large number of root-causes: lack
of atomicity~\cite{usenix-race-erickson-et-al}, unintended
sharing~\cite{race-rv-2012-talk}, or a misunderstanding of how generated code
behaves (e.g. miscompilation~\cite{Boehm:2011:MPB:2001252.2001255}).
%
Many techniques have been proposed to detect data races, either static or
dynamic.

Static race detection methods are known to provide wide coverage of the
program, since they can reason about all the inputs and the thread
interleavings, however are also known to generate many false
positives~\cite{Pratikakis:2011:LPS:1889997.1890000} and miss
races~\cite{Voung:2007:RSR:1287624.1287654}.

\begin{itemize}
\item Existing techniques
\item Happens-before
\item Lock-set
\item Hybrid
\item Pros and cons
\item Thread-sanitizer
\item Pros and cons
\end{itemize}

\subsection{Happens-Before relation}
\label{subsec:happensbefore}



%%% Local Variables:
%%% mode: latex
%%% eval: (flyspell-mode 1)
%%% TeX-master: "root.tex"
%%% End:
