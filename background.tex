\section{Background}
\label{sec:background}

Data race detection is probably one of the most widely studied of problems in
concurrent programs design and debugging, and has been shown to be
NP-hard~\cite{netzer-miller}.
%
Data races are in general a symptom for a large number of root-causes: lack
of atomicity~\cite{usenix-race-erickson-et-al}, unintended
sharing~\cite{race-rv-2012-talk}, or a misunderstanding of how generated code
behaves (e.g.\ miscompilation~\cite{Boehm:2011:MPB:2001252.2001255}).
%
Many techniques have been proposed to detect data races, either static or
dynamic.

Static race detection methods are known to provide wide coverage of the
program, since they can reason about all the inputs and the thread
interleavings, however are also known to generate many false
positives~\cite{Pratikakis:2011:LPS:1889997.1890000} and miss
races~\cite{Voung:2007:RSR:1287624.1287654}.
%
Dynamic race detection techniques analyze a specific trace of a program
that is actually executed.
%
Dynamic race detectors process the program’s events in parallel during the
execution~\cite{Lamport:1978:TCO:359545.359563, Savage:1997:EDD:269005.266641,
  Flanagan:2009, tsan}.
%
A large number of dynamic race detection tools are based on one of the
following algorithms: happens-before, lockset or both (hybrid techniques).
%
These algorithms are described in detail
in~\cite{OCallahan:2003:HDD:966049.781528}.
%
Most of dynamic data race detectors use these algorithms to detect races in
Pthreads programs.
%
In fact, existing tools, such as \tsan~\cite{tsan}, implement a very optimized
happens-before based techniques that guarantee high precision and accuracy, a
runtime overhead of $5x-20x$, and a memory overhead of $2x-10x$.
%
However, none of them is actually able to identify correctly data races in OpenMP
programs.

Even though, an OpenMP program is typically translated in a Pthreads program
by the compiler, these tools often miss races or report false positives
because cannot recognize synchronization mechanism used by the OpenMP
programming paradigm such as barriers, critical sections, etc.
%
Only existing commercial tools, such as \insp or \sun, provide data race
detection for OpenMP programs through a binary instrumentation of the
executable, and generally applying an implementation of the happens-before
relation.
%
Experiments show that such tools~\cite{Protze:2014:TPL:2688361.2688369} are
not very precise and accurate (indeed they often miss races and report many
false alarms).
%
Furthermore they generate a very high runtime and memory overhead that make
their usage often infeasible especially with large scientific HPC
applications.

%%% Local Variables:
%%% mode: latex
%%% eval: (flyspell-mode 1)
%%% TeX-master: "root.tex"
%%% End:
