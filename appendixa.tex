\section{OpenMP Concurrency Operational Semantics}
\label{sec:appendixa}

This is an excerpt of the operational semantics, the full version can be found
in~\cite{atzeni_opsem16}. \vspace{-2ex}

\subsection{Convention}
\label{subsec:convention}

\begin{compactitem}
\item $\N$ is the set of natural numbers, $\{0,1,2,\ldots\}$.
\item $t\in TID$ is a thread ID (a natural number) for some $TID\in \N$ ($TID$
  thread IDs are allowed).
\item Let $ADDR\in\N$ be the range of memory addresses accessed by the various
  threads.
\end{compactitem}

\subsection{Offset-Span Labels}
\label{subsec:osl}

We follow the concept of offset-span labels introduced in the paper of
Mellor-Crummey~\cite{Mellor-Crummey:1991:ODD:125826.125861}.
%
An offset-span label ($osl$ for short) labels each thread's execution point
with a sequence of pairs, marking its lineage in the concurrency structure
defined by prior forks and joins.
%
The domain of $OSL = ((\N\times\N))^{\N}$, i.e.\ each member $osl\in OSL$ is a
sequence of pairs
$\langle a_1,b_1\rangle, \langle a_2,b_2\rangle,\ldots,\langle a_n,b_n\rangle$
suppose $osl_1, osl_2\in OSL$.
%
These labels are sequential exactly when one of the OSLs (say $osl_1$) is a
prefix of the other.
%
Otherwise, they are concurrent.
%
For further details, please see~\cite{Mellor-Crummey:1991:ODD:125826.125861}.

\subsection{System State}
\label{subsec:systemstate}

The state of the system consists of a global states $GS$ and a set of thread
local states $TP$ (Thread Pool).
%
The total state $ts$ of any system is a pair ``Global State, Thread Pool;''
i.e., a specific total state $ts$ is:
\[ ts = \langle gs, tp \rangle \]
\noindent Each total state $ts$ comes from the domain $T$S, where
$TS = GS\times TP$.

\noindent Each global state $gs$ is a 5-tuple:
\[ \langle bm, m, n, rw, \sigma \rangle \]
\noindent Each global state $gs$ comes from the domain $GS$, where
\[ TS = BM\times M\times N\times RW\times \Sigma \]

\noindent where:
%
\begin{compactitem}
\item The domain $BM = ParRegID \mapsto (\N\times \N)$.  Thus, for each
  $bm\in BM$, we have $bm\; :\; ParRegID \mapsto (\N\times \N)$.
  %
  Given a $p\in ParRegID$, $bm$ returns a pair of natural numbers $(a,b)$,
  where $a$ is the ``current Barrier Count'' and $b$ is the ``target Barrier
  Count.''
  %
  When a thread $t$ with offset span label $osl$ executes a $ParBegin(N)$
  instruction, $N$ threads are created, and an entry
  $\langle osl, (0,N)\rangle$ is added to function $bm$\footnotemark.
  %
  \footnotetext{Recall that functions are single-valued relations, or sets of
    pairs with unique second component for each given first component.  Thus,
    $\{\langle osl,(0,N)\rangle\}$ is a function. We will allow functions to
    evolve, i.e.\ undefined for items explicitly added.}
  %
  The first field $0$ will be incremented each time we close a barrier or a
  parallel region, in a manner to be described momentarily.
  %
  Threads that have to meet a barrier or have to hit the $ParEnd$ constructs.

\item Mutex $m$ comes from domain $M$, where $M=\{-1\}\cup \N$.  Each $m$ is
  initialized to $-1$ when the mutex is free.
  %
  When thread $t\in TID$ acquires the mutex, we record the value $t$ in $m$,
  indicating that the mutex is taken, and also taken by which thread.

\item ``Nutex'' $n$ comes from domain $N$ where
  $N = Names \mapsto (\{-1\} \cup TID)$.
  %
  That is, given a named mutex name $n \in Names$, $N[n] = -1$ means that this
  named mutex (``nutex'') is free.
  %
  Otherwise, $N[n] = t$, recording the fact that this nutex is held by thread
  $t$.

\item Let memory access-type $MAT=\{R,W\}$.

\item $rw\in RW$ is a tuple that maintains all the memory accesses of each
  thread in the system.
  %
  We have
  $RW = TID\times OSL \times \N \times ADDR \times MAT \times M\times N$.
  %
  Each memory access performed by thread $t$ is recorded as the tuple
  \[ \langle tid, osl, bl, addr, mat, mutex, nutex \rangle \] where:
  \begin{compactitem}
  \item $tid\in TID$ is the thread ID;
  \item $osl\in OSL$ is the offset-span label;
  \item $bl\in \N$ is the barrier label of the last barrier seen by the thread
    $t$;
  \item $addr\in ADDR$ is the memory address;
  \item $type\in\{R,W\}$ records reads or writes;
  \item $mutex$ is the mutex state (value of $M$ in $GS$) at the time of the
    access;
  \item $nutex$ is the nutex state (value of $N$ in $GS$) at the time of the
    access.
  \end{compactitem}

\item $\sigma\in\Sigma$ is the data state of the system, as described earlier.
\end{compactitem}

\vspace{2ex}

The local state $TP$ is the thread pool that contains a list of 3-tuples, each
of which is the local state of a thread:
\[ \langle tid, osl, bl \rangle \]

The domain $TP=2^{TID\times OSL\times \N}$ where:

\begin{compactitem}
\item $t\in TID$ is thread ID of the thread;
\item $osl\in OSL$ is an offset-span label;
\item $bl\in\N$ is the label of the barrier the thread has witnessed last.
  %
  We assume that each barrier instruction is of the form $bar(L)$ where
  $L\in\N$ carries the barrier number.
  %
  A thread crossing a barrier sets its $bl$ to the value $L$.
\end{compactitem}

\subsection{Helper Functions and Predicates}
\label{subsec:helper}

\begin{compactitem}
\item $as$: is used as in Ocaml (it allows a name for a whole structure, as
  well as helps us refer to the inner details of the structure).
\item $most(lst)$: we define $most$ as a function that return the same list
  given in input except the last element (i.e.\ in Python lst[:-1]).
\item $\parallel$: This operator is used to describe that two different
  threads are concurrent.
  %
  In particular, given two offset-span labels $l1$ and $l2$, $l1 \parallel l2$
  (read $l1$ and $l2$ are concurrent) means that the two threads associated to
  the two offset-span labels are concurrent.
  %
  In case the labels represent barrier labels, it means that the two barrier
  are concurrent, in other words they happen in two different (nested)
  parallel regions.
\item $SpawnChildren(\langle ptid, posl, pbl \rangle, \sigma, N)$: Given the
  parent's TID ($ptid$), offset-span label ($posl$) and barrier label ($pbl$),
  this function creates a pool of $N$ threads -- specifically, the local
  states of these threads $\langle tid, osl, bl \rangle$.
  %
  It initializes the offset-span label $osl$ for each thread created using the
  rules in \S~\ref{subsec:osl}, by extending $posl$ with pairs $[0,N]$ through
  $[N-1,N]$.
  %
  The $bl$ is set to $pbl$.
  %
  The TIDs are somehow uniquely generated ($osl$ could be used as an index
  into an evolving TID bijection).
\item $GetChildJoin(tp)$: returns the single thread-state triple that results
  from fusing all the threads in the thread pool $tp$.
  %
  The offset-span labels are all chopped, and the single data state that is
  carried forward has the right PC and data state, going forward.
\item $Concurrent(OSL, t1, t2)$ is as described in \S\ref{subsec:osl}.
\item Function $AddRW(\langle tid, osl, bl, addr, mat, m, n \rangle)$ adds the
  access into the RW Structure.
  %
  The record says ``an access by $tid$ with offset-span label $osl$ and
  barrier label $bl$ is performed at address $addr$ with memory access type
  $mat$, when the mutex state is $m$ and the nutex state is $n$.''
\item $Full(bm, osl)$: This predicate keeps the counts the number of threads
  that have reached a $ParEnd(N)$ (or a $Barrier(bid)$) construct.
  %
  In order to count the threads, it uses the structure $bm$ which is indexed
  by the $ParRegID$ represented by the offset-span label $osl$.
  %
  In other word, the predicate $Full$ means that other threads have reached
  the construct and have incremented the counter in the $bm$ structure.
  %
  From a functional language point of view $Full$ would look like:

  \begin{center}
    \lstset{language=[Objective]Caml}
    \begin{lstlisting}
      let Full(bm, osl) =
         let (count, N) = bm[osl]
         in (count == N - 1)
    \end{lstlisting}
  \end{center}
\end{compactitem}

\subsection{Structured Operational Semantics Rules}
\label{subsec:sosrules}

\begin{framed}
  \vspace{-3ex}
  \begin{flushleft}
    \resizebox{0.95\textwidth}{!}{\
      \begin{minipage}[b]{0.5\textwidth}
        \[
        \inference[ParBegin(N)]{
          gs\; {\rm as}\; (bm, m, n, rw, \sigma) \in GS \\
          te\; {\rm as}\; (tid, osl, bl) \in TP \\
          at(tid, \sigma, ParBegin(N)) \\
          tp' = (tp - \{te\} \cup SpawnChildren(\langle tid, osl, bl \rangle, \sigma, N))\\
          bm' = bm \cup \{\langle osl,(0,N)\rangle\}\\
          \sigma' = nxt(\sigma, tid)
        }
        {\langle gs, tp \rangle \longrightarrow
          \langle gs'\; {\rm as}\; \langle bm',m,n,rw,\sigma'\rangle, tp' \rangle
        }
        \]
      \end{minipage}
      \hspace{90pt}
      \begin{minipage}[b]{0.5\textwidth}
        \[
        \inference[ParEnd(N)]{
          gs\; as\; (bm, m, n, rw, \sigma) \in GS \\
          te\; as\; (tid, osl, bl) \in TP \\
          tp' \subseteq tp \\
          at(tid, \sigma, ParEnd(N)) \\
          Full(bm, most(osl)) \\
          bm' = bm - \{\langle most(osl), * \rangle\} \\
          \sigma' = nxt(\sigma, tid) \\
          tp'' = tp - tp' \cup GetChildJoin(tp')}
        {\langle gs, tp \rangle \longrightarrow
          \langle gs'\; {\rm as}\; \langle bm',m,n,rw,\sigma'\rangle, tp'' \rangle
        }
        \]
      \end{minipage}}
  \end{flushleft}

  \begin{flushleft}
    \resizebox{0.95\textwidth}{!}{\
      \begin{minipage}[b]{0.5\textwidth}
        \[
        \inference[ParEnd(N)]{
          gs\; as\; (bm, m, n, rw, \sigma) \in GS \\
          te\; as\; (tid, osl, bl) \in TP \\
          tp' \subseteq tp \\
          at(tid, \sigma, ParEnd(N)) \\
          \neg Full(bm, most(osl)) \\
          bm[most(osl)]\; as\; (count, N) \\
          bm' = bm \cup \{\langle osl,(count+1,N) \rangle\} \\
          \sigma' = nxt(\sigma, tid)}
        {\langle gs, tp \rangle \longrightarrow \langle gs'\; {\rm as}\; \langle bm',m,n,rw,\sigma'\rangle, tp \rangle}
        \]
      \end{minipage}
      \hspace{90pt}
      \begin{minipage}[b]{0.5\textwidth}
                \[
        \inference[LoadStore]{
          gs\; as\; (bm, m, n, rw, \sigma) \in GS \\
          te\; as\; (tid, osl, bl) \in TP \\
          at(tid, \sigma, LoadStore(addr, mat)) \\
          rw' = AddRW(tid, osl, bl, addr, mat, mutex, nutex) \\
          \sigma' = nxt(\sigma, tid)}
        {\langle gs, tp \rangle \longrightarrow \langle gs'\; {\rm as}\; \langle bm,m,n,rw',\sigma' \rangle, tp \rangle}
        \]
      \end{minipage}}
  \end{flushleft}

  \begin{flushleft}
    \resizebox{0.95\textwidth}{!}{\
      \begin{minipage}[b]{0.5\textwidth}
        \[
        \inference[AcquireMutex]{
          gs\; as\; (bm, m, n, rw, \sigma) \in GS \\
          te\; as\; (tid, osl, bl) \in TP \\
          at(tid, \sigma, AcquireMutex()) \\
          m = -1 \\
          m' = tid \\
          \sigma' = nxt(\sigma, tid)}
        {\langle gs, tp \rangle \longrightarrow \langle gs'\; {\rm as}\; \langle bm,m',n,rw,\sigma' \rangle, tp \rangle}
        \]
      \end{minipage}
      \hspace{90pt}
      \begin{minipage}[b]{0.5\textwidth}
        \[
        \inference[ReleaseMutex]{
          gs\; as\; (bm, m, n, rw, \sigma) \in GS \\
          te\; as\; (tid, osl, bl) \in TP \\
          at(tid, \sigma, ReleaseMutex()) \\
          m = t \\
          m' = -1 \\
          \sigma' = nxt(\sigma, tid)}
        {\langle gs, tp \rangle \longrightarrow \langle gs'\; {\rm as}\; \langle bm,m',n,rw,\sigma' \rangle, tp \rangle}
        \]
      \end{minipage}}
  \end{flushleft}

  \begin{flushleft}
    \resizebox{0.95\textwidth}{!}{\
      \begin{minipage}[b]{0.5\textwidth}
        \[
        \inference[AcquireNutex(name)]{
          gs\; as\; (bm, m, n, rw, \sigma) \in GS \\
          te\; as\; (tid, osl, bl) \in TP \\
          at(tid, \sigma, AcquireNutex(name)) \\
          n[name] = -1 \\
          n' = n[name \rightarrow tid] \\
          \sigma' = nxt(\sigma, tid)}
        {\langle gs, tp \rangle \longrightarrow \langle gs'\; {\rm as}\; \langle bm,m,n',rw,\sigma' \rangle, tp \rangle}
        \]
      \end{minipage}
      \hspace{90pt}
      \begin{minipage}[b]{0.5\textwidth}
        \[
        \inference[ReleaseNutex(name)]{
          gs\; as\; (bm, m, n, rw, \sigma) \in GS \\
          te\; as\; (tid, osl, bl) \in TP \\
          at(tid, \sigma, ReleaseNutex(name)) \\
          n[name] = tid \\
          n' = n[name \rightarrow -1] \\
          \sigma' = nxt(\sigma, tid)}
        {\langle gs, tp \rangle \longrightarrow \langle gs'\; {\rm as}\; \langle bm,m,n',rw,\sigma' \rangle, tp \rangle}
        \]
      \end{minipage}}
  \end{flushleft}

  \begin{flushleft}
    \resizebox{0.95\textwidth}{!}{\
      \begin{minipage}[b]{0.5\textwidth}
        \[
        \inference[Barrier(bid)]{
          gs\; as\; (bm, m, n, rw, \sigma) \in GS \\
          te\; as\; (tid, osl, bl) \in TP \\
          at(tid, \sigma, Barrier(bid)) \\
          Full(bm, most(osl)) \\
          bm' = bm - \{\langle osl,* \rangle\} \\
          \sigma' = nxt(\sigma, tid)}
        {\langle gs, tp \rangle \longrightarrow
          \langle gs'\; {\rm as}\; \langle bm',m,n,rw,\sigma'\rangle, tp \rangle
        }
        \]
      \end{minipage}
      \hspace{90pt}
      \begin{minipage}[b]{0.5\textwidth}
        \[
        \inference[Barrier(bid)]{
          gs\; as\; (bm, m, n, rw, \sigma) \in GS \\
          te\; as\; (tid, osl, bl) \in TP \\
          at(tid, \sigma, Barrier(bid)) \\
          \neg Full(bm, most(osl)) \\
          bm[most(osl)]\; as\; (count, N) \\
          te' as (tid, osl, bid) \\
          tp' = tp - te \cup \{ te' \} \\
          bm' = bm \cup \{\langle osl,(count+1,N) \rangle\} \\
          \sigma' = nxt(\sigma, tid)}
        {\langle gs, tp \rangle \longrightarrow \langle gs'\; {\rm as}\; \langle bm',m,n,rw,\sigma'\rangle, tp' \rangle}
        \]
      \end{minipage}}
  \end{flushleft}

  \begin{flushleft}
    \begin{minipage}[b]{1\textwidth}
      \centering
      \resizebox{0.55\textwidth}{!}{\
        $\inference[RaceCheck]{
          gs\; as\; (bm, m, n, rw, \sigma) \in GS \\
          te_1\; as\; (tid_1, osl1, bl1) \in tp\\
          te_2\; as\; (tid_1, osl1, bl1) \in tp \\
          (tid_1 \neq tid_2) \\
          Concurrent(osl, tid_1, tid_2) \\
          i \in rw[tid_1] \\
          j \in rw[tid_2] \\
          (rw[tid_1][i].addr == rw[tid_2][j].addr) \\
          (rw[tid_1][i].mat == W) \vee (rw[tid_2][j].mat == W) \\
          (rw[tid_1][i].mutex == -1) \vee (rw[tid_2][j].mutex == -1) \\
          (rw[tid_1][i].nutex \cap rw[tid_2][j].nutex = \emptyset) \\
          (rw[tid_1][i].bl == rw[tid_2][j].bl) \vee (rw[tid_1][i].bl \parallel rw[tid_2][j].bl)}
        {\langle gs, tp \rangle \rightarrow RaceFail(\sigma, addr, tid_1, tid_2)}
        $
      }
    \end{minipage}
  \end{flushleft}
  \vspace{-2ex}
\end{framed}

%%% Local Variables:
%%% mode: latex
%%% eval: (flyspell-mode 1)
%%% TeX-master: "root.tex"
%%% End:
