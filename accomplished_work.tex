\section{Accomplished Work}
\label{sec:accomplishedwork}

The first part of the work has been accomplished and published
in~\cite{Protze:2014:TPL:2688361.2688369, atzeni2016}.

In~\cite{Protze:2014:TPL:2688361.2688369}, where I am a co-author, we present
a feasibility study of applying the static analysis and the runtime
annotations approach introduced in \S~\ref{sec:proposedresearch}.
%
We demonstrate that the static analysis through sequential blacklisting and
data dependency analysis can respectively help to exclude sequential and race
free OpenMP regions of code from the runtime analysis in order to reduce the
overhead.
%
While the runtime annotation approach can be used to make \tsan aware of
OpenMP synchronization primitives, so that it can report correctly data races
in an OpenMP program, without reporting any of all the false positives
previously and erroneously detected in the OpenMP runtime.

My main contribution is presented in the paper~\cite{atzeni2016}, which has
been accepted at IPDPS'16 and I will present in May 2016.
%
In this work we introduce \archer a data race detection for OpenMP applications.
%
\archer embeds seamlessly the static analyses for sequential blacklisting and
data dependency on \tsan compiler instrumentation mechanism as LLVM passes and
provides an annotate version of the OpenMP runtime to enable \tsan on OpenMP
programs.
%
The static analyses in synergy with the instrumentation mechanism identify the
sequential and the race free regions of code instrumenting only the rest of
the code potentially racy.
%
As a result, the compiler produces an instrumented executable linked against
the annotated OpenMP runtime provided by \archer.
%
The annotations communicate with the \tsan runtime in order to establish an
implicit happens-before relations, between the involved threads, in those
cases where the OpenMP runtime adopts synchronization primitive unknown by
\tsan (e.g. omp barrier, etc.).
%
This solution builds the knowledge of the happens-before relation into the
runtime libraries to obtain precision and accuracy in the data race detection
process.
%
As shown in~\cite{atzeni2016}, this first part of the work shows good results
in terms of reduction of runtime overhead.
%
In fact, \archer, on the OmpSCR~\cite{ompscr} and AMG~\cite{amg2013}
benchmarks, achieve a speedup of about $25\%$ while maintaining the same or
better precision and accuracy respect existing data race detectors such as
\insp and \tsan.

I have conducted and contributed to this work during our current collaboration
between our research group and a group of computer scientists at the LLNL,
where I have been a student intern in the past two summers (Summer'14 and
Summer'15).

\begin{itemize}
\item Archer Paper
\item Static + dynamic
\item Data dependency analysis
\item Sequential blacklisting
\item Current improvement in runtime and memory overhead
\end{itemize}

%%% Local Variables:
%%% mode: latex
%%% eval: (flyspell-mode 1)
%%% TeX-master: "root.tex"
%%% End:
