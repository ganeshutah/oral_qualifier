\section{Introduction}
\label{sec:introduction}

Multithreaded programming has become widespread in use, given the need to
employ multicore CPUs to gain higher performance at a given energy budget.
%
In the High Performance Computing (HPC) world, this has lead to an increased
adoption of on-node parallelism in large software applications.
%
This trend is confirmed by work that is in progress at national research
facilities~\cite{sierra, summit, trinity}.


Multithreaded programming is achieved through different programming models
(e.g. Pthreads); however, the predominant paradigm of choice in HPC is
OpenMP~\cite{ompdoc}, which guarantees portability and ease of use.
%
In fact, we are collaborating with computational scientists at the Lawrence
Livermore National Laboratory (LLNL), one of the world's largest computing
facilities, where one of the main ongoing tasks is the porting of critical
multiphysics application~\cite{llnl-apps} to OpenMP.
%
In this community, OpenMP is of paramount importance to enable shared memory
parallel programming; yet, porting large HPC applications to OpenMP is
error-prone.
%
The correctness of these applications is crucial to the reliability of
critical simulations pertaining to many real world phenomena of fundamental
importance such as modeling of nuclear explosions, weather simulations,
hydrodynamics modeling, etc.
%
One of the most common of error types in OpenMP applications is data
races~\cite{sus_common_2008}.
%
Data races are often hard to locate with traditional debugging methods.
%
Precise checking tools to detect data races are needed now more than ever.
%
While data race is a well-known problem and many Pthreads data race detection
tools have been proposed over the past twenty years, none or just a few of
them are actually able to analyze OpenMP programs.
%
Our work targets this critical need.

Data race detection research has focused both on static and dynamic analysis
techniques.
%
Static analysis techniques allow to reason about all the inputs of the program
and the interleavings of the threads, and they are fairly scalable and fast
since no runtime overhead is incurred.
%
However, the lack of information that exists only at runtime makes these
techniques very imprecise; in fact they often miss races and generate many
false positives.
%
Runtime techniques are precise, as they do not report any false positives, and
only report races in the branches of programs that are actually executed.
%
On the other hand dynamic analysis for data race detection is known to
generate a very high runtime and memory overhead due to the operations it
needs to perform and the states it needs to maintain during the execution.

The runtime overhead of even the best of dynamic tools, such as the
ThreadSanitizer (Tsan) and Intel Inspector XE, can cause between $5x-20x$
slowdown and the memory overhead can be between $2x-10x$ of the memory used by
the normal execution of the programs.
%
For very large programs, such as HPC applications, the runtime and memory
overheads can be even bigger.
%
For instance, our experiments show that this slowdown can be $100x$
and the memory consumption can be $50x$ larger.
%
The high runtime slowdown and memory usage make such tools useless from the
point of view of actual developers, who probably would not be keen on waiting
a long time to check their programs; or they may not even have enough machine
resources to run the tools.
%
We definitely need better techniques -- static, dynamic, or combinations -- to
detect data races in large structured parallel applications, while
guaranteeing precision and accuracy while incurring modest overheads.

\subsection{Thesis statement}
\label{subsec:statement}

The goal of this dissertation is to provide a sound and precise data race
detection tool for large OpenMP applications that combine static and dynamic
analysis while maintaining a low runtime and memory overhead.
% 
\emph{Our thesis statement is that by combining the best of these techniques
  and tailoring the implementation to the actual concurrency structure of
  structured parallel languages such as OpenMP, we can make data race checking
  of HPC applications practical.}

Our approach is now briefly summarized.
%
First, we combine static and existing dynamic techniques to reduce the amount
of code to analyze at runtime, thus lowering the overheads.
%
Second, we will develop a new runtime technique for data race detection that
exploits the structured parallelism of OpenMP to reduce runtime and memory
overhead, while guaranteeing high precision and accuracy.
%
Our project goals are now listed, as conceived at the beginning of our
research.

\begin{itemize}
\item \textbf{Sequential Blacklisting:} We will design a static analysis to
  identify the code executed sequentially in a OpenMP program.
\item \textbf{Data Dependency Analysis:} We will design a static analysis
  technique that identify race free code in a OpenMP program to exclude it
  from the runtime analysis.
\item \textbf{\archer v1:} We will combine the static analysis with the
  existing Clang/LLVM Tsan data race detection tool and we will make Tsan
  capable of detect data races in OpenMP programs, as part of a first version
  of a data race detector called \archer.
\item \textbf{Clock-less runtime algorithm:} We will design and implement a
  new runtime analysis technique that exploits the structured parallelism of
  the OpenMP paradigm in order to reduce the runtime and memory overheads.
\item \textbf{\archer v2:} We will embed the aforementioned new runtime
  technique in a second version of \archer as a replacement of the \tsan
  runtime.
\end{itemize}

The first three goals have been achieved in our previous
workshop~\cite{Protze:2014:TPL:2688361.2688369} and
conference~\cite{atzeni2016} papers.

%%% Local Variables:
%%% mode: latex
%%% eval: (flyspell-mode 1)
%%% TeX-master: "root.tex"
%%% End:
